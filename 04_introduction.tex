\chapter{Introduction}
\label{sec:introduction}

The idea behind this document is to find ways to overcome a specific sound design problem: In sound design for movies and the game industry, rising interest is given to synthesis techniques instead of sampling. This tendency has many reasons, one of them being that, generally speaking, a synthesized sound has different, and most of the time more, possibilities in manipulating the resulting sound. Various effects are often used in combination with one or multiple  samples as a basis for designing a certain sound to reach a certain level of artistic freedom and production-ready flexibility. Also granular synthesis or FFT analysis and re-synthesis can be used to expand the possibilities, a sample can offer as a starting point in sound design. Synthesis on the other hand can offer a more meaningful parametrization of a sound, and can provide synthesis models or programs to be re-used without sounding the same all the time. Synthesis models can be developed that are appropriate for a wider range of situations. The real problem of synthesis is that it is time consuming to come up with a proper model or an appropriate algorithm of the sound to be synthesized. \\
A very common technique for going about synthesizing an arbitrary sound is to think about what resonators and exciters might play a role. If one thinks about synthesizing a struck bar for example, one can distinguish between the force that is exciting the bar and the resonance of the bar. In consequence, one can build a digital model that resembles this behavior. Karplus-Strong algorithms are one simple example of these techniques. This method usually uses a short burst of noise to excite a delay with a feedback loop. Other techniques use a number of parallel resonant bandpass filters, which can yield very convincing and detailed results. \\

While the idea is intuitive, in practice, it is often hard to estimate the parameters of a high number of bandpass filters. Similar problems as in additive synthesis arise: While the technique is powerful in theory, the parametrization is quite a hard problem because of the partly huge number of parameters. This work therefore proposes techniques for easily extracting a parametrization for a such a system consisting of band pass filters, that can be used to synthesize sounds without needing to understand too much about the real sounding object.

