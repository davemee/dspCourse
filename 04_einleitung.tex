\chapter{Introduction}
\label{sec:introduction}

The idea behind this document is to find ways to overcome a specific sound design problem: In sound design for movies and the game industry, rising interest is given to synthesis techniques instead of sampling. This tendency has many reasons, one of them being that, generally speaking, a synthesized sound has different, and most of the time more, possibilities in manipulating the resulting sound. Various effects are often used in combination with one or multiple  samples as a basis for designing a certain sound to reach a certain level of artistic freedom and production-ready flexibility. Also granular synthesis or FFT analysis and re-synthesis can be used to expand the possibilities a sample offers as a starting point in sound design. Synthesis on the other hand can offer a more meaningful parametrization of a sound, and can provide synthesis models or programs to be re-used without sounding the same all the time and which are appropriate for a wider range of situations. \\

A very common technique for going about synthesizing an arbitrary sound is to think about what resonators and exciters might play a role. If one thinks about synthesizing a struck bar for example, one can distinguish between the force that is exciting the bar and the resonance of the bar. In consequence, on can build a digital model that resembles this behavior. Karplus-Strong algorithms are one simple example of these techniques. This method usually uses a short burst of noise to excite a delay with a feedback loop. Other techniques use a number of parallel resonant bandpass filters, which can yield very convincing and detailed results. \\

While the idea is intuitive, in practice, it is often hard to estimate the parameters of a high number of bandpass filters. Similar problems as in additive synthesis arise: While the technique is powerful in theory, the parametrization is quite a hard problem because of the partly huge number of parameters. 


A resonant band pass filter typically has three parameters: it's center frequency \(f_c\), it's input gain \(a\), and it's resonance \(q\). While all three of them need to approach the correct values for a convincing resonance model, finding the correct center frequencies for a hand full of filters is often the most challenging part. In oder to find out the resonance frequencies of a given system, various approaches can lead to a good enough approximation. First, one might calculate the resonance frequencies, which involves finding a differential equation of a system and discretizing it for example using the finite difference method, finite elements method or the binomial transform. Since this method is used in construction engineering, various CAD programs offer such capabilities for calculating resonance frequencies for a given 3d model with assigned materials. \\
Secondly, resonance frequencies of simple shapes are very easy to calculate, even by hand, such as the modes of an ideal massless string or an ideal tube. In fact, these two can be proven to be identical, meaning the idealized versions will lead to the same models \citep[see][p. 230]{cook_real_2002}. A good example of such a decomposition is the Kelly Lochbaum vocal tract model as described by \citep{smith_physical_2010}\footnote{\texttt{https://ccrma.stanford.edu/\~{}jos/pasp/Ideal\_Acoustic\_Tube.html}}. 


Since theses approaches can be very demanding and time consuming, a simpler way to model a resonance cavity is advantageous. In practice, a sound designer might just listen to a sound or analyze a recorded sound using software and spectral displays to find out what the resonance frequencies might be.\
In order to listen to or record a resonating body, it needs to be excited. Assuming that most simple resonating bodies (such as strings, tubes, plates) behave nearly linear, an impulse response of a such a system is most appropriate for analyzing it's behavior. This is the case since the impulse response, or \(H[n]\), of an LTI system is fully describing it. \
High quality impulse responses of rooms or spaces for the use of convolution reverbs are typically obtained using exponential sine sweeps(ESS). In case of a body, instead of a room, the problem of transducing a broad range of frequncy components using a transducer arises. Therefore in practice, impulse responses of bodies are often obtained by hitting the body:

\begin{quotation}
\textit{"A small gas lighter or “clacker” used for some sound-controlled toys is useful. This is great for capturing the interior of vehicles or similar spaces. Other times you simply want to tap the outside of a rigid body to get an impression of its material makeup and impact sound. With either the impulse source or the object you use for tapping out excitations, it’s important to use the same thing consistently. A drumstick with plastic coated tip is ideal since it has a reasonable weight, is easily replaceable, and won’t dent or damage anything."}
\end{quotation}
\citep[p. 230]{farnell_designing_2010}\

In theory, such a recording of an impulse response of a body should be recorded in an anechoic chamber, otherwise the room in which the sound was recorded will also come into play. In practice, this will rather not be the case, so this paradigm will also also be ignored in this work. Also, a logical next step to use an impulse response one obtained is to use it to convolve an excitation with it rather than trying to analyze it. On the other hand, convolution can be a very CPU intense process and is not very easy to manipulate. It is far easier t control the decay time of some number of bandpass filters than to control the decay of convolution for instance.\

This work therefore tries to find and describe methods for the estimation of parameters of resonance filters based on such low quality recordings of impulse response-like signals of some example bodies and tries to analyze the strengths and weaknesses of the results.





