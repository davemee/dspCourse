\section{Finding Gains and Decay Rates}
\label{sec:gainsAndDecays}

Once the loudest frequencies have been found, getting the gains \(a_0 ... a_i\) ,  and decay rates \(\alpha_0 ... \alpha_i\) can be very simple. The most straight forward approach is just choosing two frames of the spectrum, using the first one to determine the gain relationships and comparing it to one recorded later in time to find out the decay rates. \\
The decay of the components is assumed to follow an exponential law, so:
\begin{align}
	a_2 = a_1 e^{-t\alpha}
\end{align}
Where \(t\) is the time, \(a_1\) is the starting point of the decay, \(a_2\) is the end point and \(\alpha\) is the rate of decay. Therefore, given two arbitrary points in time of a decaying function, the decay rate can be determined using:

\begin{align}
	\label{eq:computeAlpha}
	\alpha = \frac {\ln(\frac{a_2}{a_1})} {\Delta t}
\end{align}
In the Max implementation, the user can choose a starting-frame and an end-frame and for each detected frequency the amplitudes of these two frames are compared. One problem arises with this simple approach, though: Various effects such as amplitude modulation-like dispersion effects, see \citep[p. 445]{farnell_designing_2010}, FFT errors and non-linearities can cause a certain frequency to seemingly rise up over a short period of time or seemingly undergo tremolo-like modulation. In this case, equation \ref{eq:computeAlpha} will of course produce faulty results. A solution to this problem is to slightly filter the spectrum in it's course over time (instead of only smoothing in the frequency domain) and choosing a very \glqq{}late\grqq{} end-frame.\\

