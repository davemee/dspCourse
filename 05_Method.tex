\chapter{Method}
\label{sec:Method}





A resonant band pass filter typically has three parameters: it's center frequency \(f_c\), it's input gain \(a\), and it's resonance \(q\). While all three of them need to approach the correct values for a convincing resonance model, finding the correct center frequencies for a hand full of filters is often the most challenging part. In oder to find out the resonance frequencies of a given system, various approaches can lead to a good enough approximation. First, one might calculate the resonance frequencies, which involves finding a differential equation of a system and discretizing it for example using the finite difference method, finite elements method or the binomial transform. Since this method is used in construction engineering, various CAD programs offer such capabilities for calculating resonance frequencies for a given 3d model with assigned materials. \\
Secondly, resonance frequencies of simple shapes are very easy to calculate, even by hand, such as the modes of an ideal massless string or an ideal tube. In fact, these two can be proven to be identical, meaning the idealized versions will lead to the same models \citep[see][p. 230]{cook_real_2002}. A good example of such a decomposition from a complex shape into a number of simple shapes is the Kelly Lochbaum vocal tract model as described by \citep{smith_physical_2010}\footnote{\texttt{https://ccrma.stanford.edu/\~{}jos/pasp/Ideal\_Acoustic\_Tube.html}}. 


Since these approaches can be very demanding and time consuming, a simpler way to model a resonance cavity is advantageous. In practice, a sound designer might just listen to a sound or analyze a recorded sound using software and spectral displays to find out what the resonance frequencies might be.\
In order to listen to or record a resonating body, it needs to be excited. Assuming that most simple resonating bodies (such as strings, tubes, plates) behave nearly linear, an impulse response of a such a system is most appropriate for analyzing it's behavior. This is the case since the impulse response, \(h[n]\), of an LTI system is fully describing it. \
High quality impulse responses of rooms or spaces for the use of convolution reverbs are typically obtained using exponential sine sweeps(ESS). In case of a body, instead of a room, the problem of transducing a broad range of frequncy components using a transducer arises. Therefore in practice, impulse responses of bodies are often obtained by hitting the body:

\begin{quotation}
\textit{"A small gas lighter or “clacker” used for some sound-controlled toys is useful. This is great for capturing the interior of vehicles or similar spaces. Other times you simply want to tap the outside of a rigid body to get an impression of its material makeup and impact sound. With either the impulse source or the object you use for tapping out excitations, it’s important to use the same thing consistently. A drumstick with plastic coated tip is ideal since it has a reasonable weight, is easily replaceable, and won’t dent or damage anything."}
\end{quotation}
\citep[p. 252]{farnell_designing_2010}


In theory, such a recording of an impulse response of a body should be recorded in an anechoic chamber, otherwise the room in which the sound was recorded will also come into play. In practice, this will rather not be the case, so this paradigm will be ignored in this work. Also, a logical next step, once an an impulse response is obtained, is to use it to convolve an excitation signal with it, rather than trying to analyze it. On the other hand, convolution can be a very CPU intense process and is not very easy to manipulate. It is far easier to control the decay time of some number of bandpass filters than to control the decay of convolution for instance, but also manipulating the frequency content is considerably easier using resonant bandpass filters.\

This work therefore tries to find and describe methods for estimating parameters of resonance filters based on such low quality recordings of impulse response-like signals of some example bodies and tries to analyze the strengths and weaknesses of the results. For this purpose Python and Max/MSP are used to analyze and synthesize sounds.\\
The combination of Max/MSP and Python are ideal for the given task. Max/MSP is known to be a very fast way for the implementation of DSP related ideas that are supposed to work in real time. Another advantage of Max/MSP is how easily GUIs can be built. Python on the other hand has every comfort of a modern object oriented, open source programming language. Analysis and plotting of data is much more convenient inside python, mainly since libraries are available for analyzing and processing data in scientific ways. Maybe the most important one of these is the scientific computing library Scipy\footnote{\texttt{http://www.scipy.org/}}. Scipy includes packages for working in a Matlab similar way inside python, for plotting, for data analysis and for enhanced interactive control of the python shell. Tools used in this document include matplotlib (plotting), numpy (algebra, arrays etc.), ipython (interactive python environment), sympy (symbolic mathematics). The written code is finally presented here through the ipython notebook\footnote{\texttt{http://ipython.org/notebook.html}}, which allows in-line plotting and has become the de facto standard for presenting python code. The ipython notebook allows for export of python code to \LaTeX \ which has been used to write this document.\\
Finally Max/MSP and Python interact using two externals written by Thomas Grill\footnote{\texttt{http://grrrr.org/}}. These externals allow python code to run inside Max/MSP and to be controlled via Max. In this case, lists are handed over from Max to Python, and lists are received in return. Python code that is supposed to run inside Grill's externals needs to follow certain conventions, which is the reason for the presence of the code in appendix \ref{pythonMaxCode}.



