% \chapter{Introduction}
% \label{introduction}

\addsec{Introduction/Notes}
\label{introduction}


Umstrukturieren!
Evtl folgende Struktur dies semester:

iber: wavetable
dann
\begin{enumerate}
	\item Sampling, Waveshaping, non-linearity, signal metering
	\item AM, FM, sounddesign challenge. evtl. filter Anfang
	\item Filter + spiel
	\item waveguides, PM
	\item FFT

\end{enumerate}


Pd muss auch vermittelt werden, nicht direkt audio/DSP.
Im nächsten Semester aufgaben von hier mit reinnehmen!
\url{
http://www.soundonsound.com/sos/allsynthsecrets.htm
}
AM/FM fehlt?
vorbereiten:
\begin{itemize}
	\item FFT händisch rechnen.
	\item umstrukturierung: sampling an den anfang. Audio überblicks-übung.
\end{itemize}

\addsec{Form}
\addsec{Inverted Classroom}
\label{sub:inverted_classroom}

Nicht erledigen d. Hausübung bzw. nicht anschaun d. videos macht das konzept kaputt. Macht unsere stunden zum teil sinnlos.
Konzept soll vorteile für studenten und vortragende bringen!

% \subsection{Dialoghaftigkeit}
% \label{sub:dialoghaftigkeit}



\addsec{Sinnfrage d. Inhalts/unterrichts klären}

\addsec{Unterscheidung: Tool lernen, technik/theorie lernen}